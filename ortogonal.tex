\section{Ortogonalidade}

\subsection*{Vetores Ortogonais}

\begin{frame}[label=orto]{Vetores Ortogonais}
\begin{exer}
Considere os vetores $\vec{v}_1=(2,1,-1)$, $\vec{v}_2=(1,0,2)$ e $\vec{v}_3=(2,-5,-1)$. 

\begin{enumerate}
\item Mostre que os três vetores são ortogonais dois a dois.

\item Use esse fato para mostrar que os vetores são LI, isto é, a equação vetorial 
\[x\vec{v}_1+y\vec{v}_2+z\vec{v}_3=\vec{0},\]
{\color{red}só possui} a solução trivial $x=y=z=0$.

\item Escreva  $\vec{v}=(3,-1,1)$ como combinação linear da base $\mathcal{B}=\{\vec{v}_1,\vec{v}_2,\vec{v}_3\}$.
\end{enumerate}
\end{exer}

\end{frame}

\begin{frame}[label=orto]{}
\begin{defin}
Um conjunto de vetores $\{\vec{v}_1,\vec{v}_2,\ldots, \vec{v_k}\}\subset \R^n$ é dito {\color{blue}ortogonal} se todos os vetores são ortogonais dois a dois, isto é,
\[\vec{v}_i\cdot \vec{v}_j=0,\ \forall i,j=1,2,\ldots k, \ i\neq j.\]
E dizemos que {\color{blue}ortonormal} se além de ortogonal cada vetor for unitário, isto é, 
\[\|\vec{v}_i\|=1,\ \forall i=1,2,\ldots,k.\]
Neste caso, escrevemos também
\[
\vec{v}_i\cdot \vec{v}_j=\delta_{ij}=
\begin{cases}
1, & i=j\\ 0, & i\neq j
\end{cases}, \ \forall i,j=1,2,\ldots,k
\]
onde $\delta_{ij}$ é o chamado {\color{blue}Delta de Kronecker}.
\end{defin}

\end{frame}

\begin{frame}[label=orto]{}
%\begin{teo}
Se $\{{\color{blue}\vec{v}_1,\vec{v}_2,\ldots, \vec{v}_k}\}$  de $\R^n$  é ortogonal, então esses vetores são LI. Além disso, se
\begin{equation}\label{eq-proj}
{\color{red}\vec{v}}=c_1{\color{blue}\vec{v}_1}+c_2{\color{blue}\vec{v}_2}+\cdots +c_k{\color{blue}\vec{v}_k},
\end{equation}
então
\[c_i=\frac{{\color{red}\vec{v}}\cdot{\color{blue} {\color{blue}\vec{v}_i}}}{\|{\color{blue}\vec{v}_i}\|^2}\] 
%\end{teo}

\begin{defin}
Definimos a {\color{blue}projeção ortogonal} de um {\color{red}$\vec{v}$} sobre um vetor  {\color{red} não nulo} {\color{blue}$\vec{w}$} como sendo o seguinte vetor
\[\proj_{{\color{blue}\vec{w}}}{\color{red}\vec{v}}=
\left(\frac{ {\color{red}\vec{v}}\cdot
{\color{blue}\vec{w}} }{\|{\color{blue}\vec{w}}\|^2}\right)
{\color{blue}\vec{w}}.
\]
\end{defin}
Com esta notação, temos que a identidade \eqref{eq-proj} se escreve como
\[{\color{red}\vec{v}}=\proj_{{\color{blue}\vec{v}_1}}{\color{red}\vec{v}}+\proj_{{\color{blue}\vec{v}_2}}{\color{red}\vec{v}}+\cdots +\proj_{{\color{blue}\vec{v}_k}}{\color{red}\vec{v}}.\]
\end{frame}

\begin{frame}[label=orto]{}

\begin{exe}
Mostre $\vec{v}_1=(0,2,0)$, $\vec{v}_2=(3,0,3)$ e $\vec{v}_3=(-4,0,4)$ formam uma base ortgonal de $\R^3$. Expresse $\vec{v}=(1,2,4)$ como uma combinação linear deles. 
\end{exe}

\end{frame}


%\begin{frame}[label=orto]{Processo de Ortogonalização de Gram-Schimidt}
%\begin{prop}
%Seja $\{\vec{v}_1,\vec{v}_2,\ldots, \vec{v}_k\}$ um conjunto ortogonal  de $\R^n$. Então, para qualquer $\vec{v}\in \R^n$, temos que
%\[\vec{v}-\proj_{{\vec{v}_1}}\vec{v}-\proj_{{\vec{v}_2}}\vec{v}-\cdots-\proj_{{\vec{v}_k}}\vec{v}\]
%é ortogonal a cada $\vec{v}_i$, onde $i=1,2,\ldots,k$.
%\end{prop}
%
%\begin{exe}
%Determine uma base ortogonal para o subespaço de $\R^5$ gerado por
%\[\vec{v}_1=(-1,0,-1,0,1),\ \vec{v}_2=(0,0,1,1,0),\ \vec{v}_3=(-1,1,0,0,0).\]
%\end{exe}
%\end{frame}


\subsection*{Processo de ortonalização de Gram-Schmidt}
%\begin{frame}[label=orto]{Processo de ortonalização de Gram-Schmidt}
%\begin{prop}
%Seja $\{\vec{w}_1,\vec{w}_2,\ldots, \vec{w}_k\}$ um {\color{blue}conjunto 
%ortogonal}  de $\R^n$. Então, para qualquer ${\color{red}\vec{v}}\in \R^n$,
% temos que
%\[\vec{w}={\color{red}\vec{v}}-\proj_{{\vec{w}_1}}{\color{red}\vec{v}}
%-\proj_{{\vec{w}_2}}{\color{red}\vec{v}}-\cdots-\
%\proj_{{\vec{w}_k}}{\color{red}\vec{v}}\]
%é ortogonal a cada $\vec{w}_i$, onde $i=1,2,\ldots,k$. Além disso, $\vec{w}$ 
%está no mesmo espaço gerado por $\{\vec{w}_1,\vec{w}_2,\ldots, \vec{w}_k,{\color{red}\vec{v}}\}$.
%\end{prop}
%
%
%\begin{exe}
%Sejam $W$ o subespaço gerado por $\vec{v}_1=(1,-1,-1,1)$ e $\vec{v}_2=(2,1,0,1)$. Obtenha uma base ortogonal para $W$.
%\end{exe}
%\end{frame}


\begin{frame}[label=orto]{Processo de ortonalização de Gram-Schmidt}
Seja $\{v_1,v_2,\ldots, v_k \}$ uma base para um subespaço $W$ de $\R^n$. 
Podemos obter uma base ortogonal para $W$ através do seguinte processo, conhecido como {\color{blue}Processo de Ortogonalização de Gram-Schmidt}.
\begin{align*}
& \vec{w}_1=\vec{v}_1,\\
& \vec{w}_2=\vec{v}_2-\proj_{{\vec{w}_1}}\vec{v}_2,\\
& \vec{w}_3=\vec{v}_3-\proj_{{\vec{w}_1}}\vec{v}_3
-\proj_{{\vec{w}_2}}\vec{v}_3,\\
& \vdots\\
& \vec{w}_k=\vec{v}_k-\proj_{{\vec{w}_1}}\vec{v}_k
-\proj_{{\vec{w}_2}}\vec{v}_k-\cdots-\proj_{{\vec{w}_{k-1}}}\vec{v}_k
\end{align*}


\begin{exe}
Determine uma base ortogonal para o subespaço de $\R^5$ gerado por
\[\vec{v}_1=(-1,0,-1,0,1),\ \vec{v}_2=(0,0,1,1,0),\ \vec{v}_3=(-1,1,0,0,0).\]
\end{exe}

\end{frame}




\begin{frame}[label=orto]{}

\begin{casa}
Determine uma base ortogonal para o subespaço de $\R^4$ gerado por
\[\vec{v}_1=(1,-1,-1,1),\ \vec{v}_2=(2,1,0,1),\ \vec{v}_3=(2,2,1,2).\]
\end{casa}
\end{frame}



\section{Diagonalilzação de Matrizes Simétricas}


\subsection*{Matrizes Ortogonais}
\begin{frame}[label=orto]{Matrizes Ortogonais}

\begin{defin}
Uma matriz quadrada $Q$ cujas colunas formam um conjunto ortonormal é chamada 
de \dt{matriz ortogonal}.
\end{defin}

\begin{teo}
Seja $Q$ uma matriz $m\times n$. Suas colunas formam um conjunto ortogonal se, 
e somente se, $Q^TQ=I_n$. Em particular, $Q$ é ortogonal se, e somente se, $Q^{-1}=Q^T$.
\end{teo}

\begin{exe}
Mostre que  a seguinte matriz de rotação é ortogonal.
\[R_\theta= 
\begin{bmatrix}
\cos(\theta) & -\sen(\theta)\\
\sen(\theta) & \cos(\theta)
\end{bmatrix}
\]
\end{exe}

\end{frame}


\begin{frame}[label=orto]{}

\begin{exe}
Mostre que  a seguinte matriz é diagonalizável
\[A=
\begin{bmatrix}
1& 2\\ 2 & -2
\end{bmatrix}.
\]
Em seguida, obtenha $Q$ ortogonal que diagonalize $A$.
\end{exe}

\begin{defin}
Uma matriz quadrada $A$ é \dt{ortogonalmente diagonalizável} se é 
diagonalizável e a matriz $Q$ que a diagonaliza é ortogonal.
\end{defin}

\begin{prop}
Se $A$ é uma matriz simétrica, então os autovetores associados a autovalores distintos são ortogonais.
\end{prop}

\end{frame}

\subsection*{Teorema Espectral}

\begin{frame}[label=orto]{}

\begin{block}{Teorema Espectral}
Uma matriz  $A$ quadrada com entradas reais  é ortogonalmente diagonalizável se, e somente se, $A$ é simétrica.
\end{block}

\begin{exe}
Encontre  uma diagonalização ortogonal para a matriz
\[A=
\begin{bmatrix}
1 & 2 & 1\\ 1& 2 & 1 \\ 1 & 1 & 2
\end{bmatrix}
\]
\end{exe}

\end{frame}
